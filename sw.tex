\section{Computational infrastructure}

\begin{frame}{}
  \begin{center}
    \Large{\textbf{Behind the scenes}: software/data structures and
      open research practice.}
  \end{center}
\end{frame}


\begin{frame}{}

  Beyond the figures\footnote{... which are all reproducible, by the way.}

  \begin{itemize}
  \item<+-> Software: \textbf{infrastructure}
    (\href{http://bioconductor.org/packages/MSnbase}{\texttt{MSnbase}},
    \cite{Gatto:2012}), \textbf{dedicated machine learning}
    (\href{http://bioconductor.org/packages/pRoloc}{\texttt{pRoloc}},
    \cite{Gatto:2014a}), \textbf{interactive
      visualisation}\footnote{\url{https://lgatto.shinyapps.io/christoforou2015/}}
    (\href{http://bioconductor.org/packages/pRolocGUI}{\texttt{pRolocGUI}},
    \cite{pRolocGUI}) and \textbf{data}
    (\href{http://bioconductor.org/packages/pRolocdata}{\texttt{pRolocdata}},
    \cite{Gatto:2014a}) for spatial proteomics.
  \item<+-> The \href{http://bioconductor.org/}{\textbf{Bioconductor}}
    \citep{Huber:2015} ecosystem for high throughput biology data
    analysis and comprehension: \textbf{open source}, and
    \textbf{coordinated and collaborative\footnote{between and within
        domains/software} open development}, enabling
    \textbf{reproducible research}, enables understanding of the data
    (not a black box) and \textbf{drive scientific innovation}.
  \end{itemize}
\end{frame}


%% \begin{frame}
%%   \begin{figure}[h]
%%     \centering
%%     \includegraphics[width=.8\linewidth]{./figs_all/g.png}
%%     \caption{\textbf{Collaboration between packages}: Dependency graph
%%       containing 41 MS and proteomics-tagged packages (out of 100+)
%%       and their dependencies. }
%%   \end{figure}
%% \end{frame}

%% \begin{frame}{\textbf{MSnbase} example}

%%   \begin{figure}[h]
%%     \centering
%%     \includegraphics[width=.8\linewidth]{./figs_all/msnbase-contributors-2.png}
%%     \caption{\textbf{Collaboration within packages}: Contributions to the
%%       \texttt{MSnbase} package (1220 downloads from unique IP
%%       addresses in January 2018) since its creation, the last one
%%       leading to \textbf{common proteomics/metabolomics
%%         infrastructure}. More details:
%%       \url{https://lgatto.github.io/msnbase-contribs/}}
%%     \label{fig:msnbase}
%%   \end{figure}

%% \end{frame}

\begin{frame}{Open research: open source software}
  \centering
  \begin{figure}
  \includegraphics[width=\linewidth]{./figs_all/pRoloc_screen.png}
    \caption{\cite{Gatto:2014} Left: Public repository for the \texttt{pRoloc} software
      (\url{https://github.com/lgatto/pRoloc}). Right: offical
      Bioconductor page.}
  \end{figure}
\end{frame}

\begin{frame}{Open and reproducible research}
  \centering
  \begin{figure}
    \includegraphics[width=1\linewidth]{./figs_all/qsep_screen.png}
    \caption{\cite{Gatto:2018} reproducible document
      (\url{https://github.com/lgatto/QSep-manuscript}), preprint
      (\url{https://doi.org/10.1101/377630}) and paper
      (\url{https://doi.org/10.1016/j.cbpa.2018.11.015}).}
  \end{figure}
\end{frame}

\begin{frame}{}
Working with open and reproducible research in mind doesn't mean
releasing everything prematurely, it means

\begin{itemize}
\item managing research in a way one can find data and results at
  every stage

\item one can reproduce results, re-run/compare them with new data or
  different methods/parameters, and

\item  one can release data (or parts thereof) when/if appropriate.
\end{itemize}
\end{frame}

%% \begin{frame}{\texttt{MSnSet} data structure}

%%   \begin{figure}[h]
%%     \centering
%%     \includegraphics[width=.8\linewidth]{./figs_all/msnset.png}
%%     \label{fig:msnset}
%%   \end{figure}

%% \end{frame}


%% \begin{frame}[fragile]
%% <<spatprot0, eval=TRUE, message=FALSE, warning=FALSE>>=
%% library("pRoloc")
%% library("pRolocdata")
%% data(hyperLOPIT2015)

%% setStockcol(paste0(getStockcol(), 80))

%% library("magrittr")
%% library("dplyr")
%% library("ggplot2")
%% @
%% \end{frame}

%% \begin{frame}[fragile]
%% <<spatprot1, eval=FALSE>>=
%% plot2D(hyperLOPIT2015, fcol = NULL,
%%        col = "#00000025", pch = 19)
%% plot2D(hyperLOPIT2015, method = "hexbin")
%% plot2D(hyperLOPIT2015)

%% plot2D(hyperLOPIT2015, fcol = "final.assignment")
%% sz <- exp(fData(hyperLOPIT2015)$svm.score) - 1
%% plot2D(hyperLOPIT2015, fcol = "final.assignment",
%%        cex = sz)
%% addLegend(hyperLOPIT2015)
%% @
%% \end{frame}

%% \begin{frame}[fragile]
%%   \centering
%% <<spatprot1eval, out.width='.8\\linewidth', eval=TRUE, echo=FALSE, fig.width = 12, fig.height = 12>>=
%% par(mfrow = c(2, 2), mar = c(4, 2, 0, 0))
%% plot2D(hyperLOPIT2015, fcol = NULL,
%%        col = "#00000025", pch = 19)
%% ## plot2D(hyperLOPIT2015, method = "hexbin")
%% plot2D(hyperLOPIT2015)

%% plot2D(hyperLOPIT2015, fcol = "final.assignment")
%% sz <- exp(fData(hyperLOPIT2015)$svm.score) - 1
%% plot2D(hyperLOPIT2015, fcol = "final.assignment",
%%        cex = sz)
%% addLegend(hyperLOPIT2015)
%% @
%% \end{frame}

%% \begin{frame}[fragile]
%%   \tiny
%% <<spatprot2>>=
%% unknownMSnSet(hyperLOPIT2015) %>%
%%     fData %>% select(final.assignment) %>% table
%% @
%% \end{frame}

%% \begin{frame}[fragile]
%%     \tiny
%% <<spatprot3, fig.width = 10, fig.heigth = 6>>=
%% unknownMSnSet(hyperLOPIT2015) %>% fData %>%
%%     select(final.assignment, svm.score) %>%
%%     ggplot(aes(x = final.assignment, y = svm.score)) +
%%     geom_boxplot(aes(fill =  final.assignment))
%% @
%% \end{frame}
